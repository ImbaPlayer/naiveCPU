% File: intro.tex
% Date: Mon Oct 28 00:35:18 2013 +0800
% Author: 

\section{实验目的}

\section{实验工具及环境}
\begin{description}
  \item[操作系统] Windows XP (虚拟机)
  \item[软件] Xilinx ISE 14.6
  \item[语言] Verilog
\end{description}

\section{实验原理}

\section{模块设计}

各模块详细代码见附录.

\section{实验过程}
首先测试发现三个人的Linux系统上的Xilinx软件都无法正常连接实验平台,
于是开始安装Windows XP虚拟机并安装Xilinx, 之后开始代码调试.

代码烧入后实验机上没有任何反应, 经过分析后认为应该是Reset的状态不对.
通过修改代码输出Reset值, 发现Reset按下为0, 与代码期待行为相反, 因而实验平台一直处于Reset状态.

之后的主要问题是不清楚平台上管脚高低位的对应顺序, 因此无法观测程序行为.
做了几次小实验后清楚了平台设置, 再将原始代码烧入后行为便正常了.
做了一些测试后均未出现问题, 于是结束实验.

\section{经验及总结}
这次实验总体比较简单, 我们在实验前期主要复习巩固了verilog语法,
也熟悉了仿真软件的使用.
经过仿真, 可以尽早消灭bug, 大大减少实验中的调试时间.

实验中在环境配置上出了些问题, 但也很快通过虚拟机得到了解决.

另外, 这次的实验, 我们将各部分功能进行了模块化设计,
锻炼了初步的模块设计能力, 希望能为今后进一步的实验有所帮助.

\section{思考题}
\begin{enumerate}
  \item ALU所使用的电路是组合逻辑电路还是时序逻辑电路?

    因为输入机拥有四种状态,其行为与所处状态有关,因此为时序逻辑电路。

  \item 如果给定了A和B的初值, 每次运算后都把结果写入B中再进行下次运算,
    这样的一个带暂存功能的ALU要增加一些什么电路来实现?

    给定了A和B的初值需要两个可读寄存器,在进入输入阶段时先进行赋值。
    结果写入B说明需要用寄存器存下运算的结果,然后再输入阶段再次赋值给B。
    具体的电路实现可以使用D触发器进行实现。

\end{enumerate}

\section{实验代码}

\subsection{存储器}

\versrc{src/Mem/inputState.v}
\versrc{src/Mem/core.v}
\versrc{src/Mem/selector.v}
\versrc{src/Mem/adder.v}
\versrc{src/Mem/mem.v}

\subsection{串口访问}

%\versrc{src/SerialPort/serial\_port.v}
