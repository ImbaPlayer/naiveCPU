% File: intro.tex
% Date: Mon Oct 28 00:35:18 2013 +0800
% Author: Yuxin Wu <ppwwyyxxc@gmail.com>

\section{实验目的}
\begin{enumerate}
  \item 熟悉硬件描述语言及开发环境,了解硬件系统开发的基本过程
  \item 掌握ALU基本设计方法和简单运算器的数据传送通路
  \item 验证ALU的功能
\end{enumerate}


\section{实验工具及环境}
\begin{description}
  \item[操作系统] Windows XP (虚拟机)
  \item[软件] Xilinx ISE 14.6
  \item[语言] Verilog
\end{description}

\section{实验原理}
本实验内容是根据算术逻辑运算单元的功能表,通过状态机状态的变化,达到改变控制信号
的组合的目的,从而实现不同的算术与逻辑运算功能,并将结果与标志位显示出
来.实验中的ALU可以实现基本的算术运算、逻辑运算、移位运算等,功能如下表所示:

\begin{table}[H]
\begin{center}
\begin{tabular}{|c|c|c|}
\shline
操作码 & 功能 & 描述\\ \shline
ADD & A+B & 加法\\ \hline
SUB & A-B & 减法\\ \hline
AND & A and B & 与\\ \hline
OR & A or B & 或\\ \hline
XOR & A xor B & 异或\\ \hline
NOT & not A & 非\\ \hline
SLL & A sll B & 逻辑左移\\ \hline
SRL & A sll B & 逻辑右移\\ \hline
SRA & A sra B & 算术右移\\ \hline
ROL & A rol B & 循环左移\\ \shline
\end{tabular}
\end{center}
\end{table}

\section{模块设计}
顶层alu分为三个模块:
\begin{enumerate}
  \item 状态机inputState: 接收clk, 在读入/输出的四种状态中切换, 并向core和selector发送指令.
  \item 计算核心core: 当状态机状态进入计算时, 对输入数据执行计算.
  \item 选择器selector: 当状态机进入某个输出状态时, 选择相应的flag或result进行输出.
\end{enumerate}
相关接口如\verb|alu.v|所示:
\versrc{src/alu.v}

各模块详细代码见附录.

\section{实验过程}
首先测试发现三个人的Linux系统上的Xilinx软件都无法正常连接实验平台,
于是开始安装Windows XP虚拟机并安装Xilinx, 之后开始代码调试.

代码烧入后实验机上没有任何反应, 经过分析后认为应该是Reset的状态不对.
通过修改代码输出Reset值, 发现Reset按下为0, 与代码期待行为相反, 因而实验平台一直处于Reset状态.

之后的主要问题是不清楚平台上管脚高低位的对应顺序, 因此无法观测程序行为.
做了几次小实验后清楚了平台设置, 再将原始代码烧入后行为便正常了.
做了一些测试后均未出现问题, 于是结束实验.

\section{经验及总结}
这次实验总体比较简单, 我们在实验前期主要复习巩固了verilog语法,
也熟悉了仿真软件的使用.
经过仿真, 可以尽早消灭bug, 大大减少实验中的调试时间.

实验中在环境配置上出了些问题, 但也很快通过虚拟机得到了解决.

另外, 这次的实验, 我们将各部分功能进行了模块化设计,
锻炼了初步的模块设计能力, 希望能为今后进一步的实验有所帮助.

\section{思考题}
\begin{enumerate}
  \item ALU所使用的电路是组合逻辑电路还是时序逻辑电路?

    组合逻辑电路

  \item 如果给定了A和B的初值, 每次运算后都把结果写入B中再进行下次运算,
    这样的一个带暂存功能的ALU要增加一些什么电路来实现?


\end{enumerate}


\section{实验代码}
\versrc{src/inputState.v}
\versrc{src/core.v}
\versrc{src/selector.v}
